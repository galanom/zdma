\cleardoublepage
\rm
\begin{center}
	\textcolor{gray} {\Large
		Efficient Support for Partially Reconfigurable Accelerators\\
		in FPGA SoC for the GNU/Linux Operating System\\
	}

	\vspace{20pt}
	\textsc{\large Abstract}\\
	\vspace{0pt}
\end{center}

Currently there is a rising trend in accelerating specialized computation,
mostly driven by the growing need for machine learning. 
The solutions can be classified to two groups: 
novel processor architectures, tailored for the targeted computation problem,
and hardware acceleration, which is represented mostly by the FPGAs due
to their capability to adapt or re-purposed to a different enviornment.

Partial reconfiguration further extends the FPGA flexibility
by allowing the reconfiguration of a part of the device
during run-time without interrupting the overall system operation.
This technology makes possible to create a system that offers mutliple accelerator cores
that can be reconfigured on-demand during normal system operation.

This work implements such a system using the Zynq SoC from Xilinx. It consists of the following parts:

\begin{itemize}
\item	Hardware implementation of one homogeneous low-latency
	and one heterogeneous high-throughput 
	accelerator system for Zynq-7000 SoC,
	as well as one homogeneous and balanced system for an UltraScale+ SoC.
\item	A Linux device driver supporting any system design under its specifications.
\item	A system library that provides a user-friendly API for managing the accelerators.
\item	An application in image processing that implements some common accelerators.
\end{itemize}

In this modular system, each component is isolated and provides an abstracted interface to the others. 
At the user level, the system provides a simple API that hides all hardware details.
Using this API, the user can request a computation from the system which will be scheduled for execution
when a hardware resource is available.
The system administrator may add, remove or restrict the accelerator availablity to system slots or
may configure the behavior the scheduler and modify security policies, 
all without interrupting the normal operation of the system.

Finally, the system takes advantage of the system interconnect to provide 
concurrent data transfers and parallel access to memory, maximizing total I/O throughput.
In order to permit the exploration for the optimal design,
we made the system flexible enough to accept any accelerator arrangement 
and any memory and interconnect layout without any software modification.
\vspace*{\fill}

